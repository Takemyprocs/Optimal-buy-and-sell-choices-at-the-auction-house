\documentclass[12pt]{article}         % the type of document and font size (default 10pt)
\usepackage[margin=1.0in]{geometry}   % sets all margins to 1in, can be changed
\usepackage{moreverb}                 % for verbatimtabinput -- LaTeX environment
\usepackage{amssymb}                  % for many mathematical symbols
\usepackage[pdftex]{lscape}           % for landscaped tables
\usepackage{longtable}                % for tables that break over multiple pages
\usepackage{blkarray}
\usepackage{amsmath}

\title{The Buyer-Seller Problem}
\author{Takemyprocs}
\usepackage{Sweave}
\begin{document}
\Sconcordance{concordance:framework.tex:framework.Rnw:%
1 11 1 1 0 69 1 1 45 10 0 1 2 7 1}


\maketitle

\section{The Seller Problem: a basic framework}

Suppose a seller is endowed with a fixed amount of $m$ inputs, namely: $s_1, s_2,...,s_m$. With them the seller can craft $n$ products, namely: $x_1,x_2,...,x_n$. Not all inputs may be required for crafting a particular product, so we define $A$ as the technical matrix that captures the input requirements for each output:

\[
A = 
\begin{blockarray}{ccccc}
& x_1 & x_2 & ... & x_n \\
\begin{block}{c(cccc)}
  s_1 & a_{11} & a_{12} & ...    & a_{1n} \\
  s_2 & a_{21} & a_{22} & ...    & a_{2n} \\
  ... & ...    & ...    & \ddots & ... \\
  s_m & a_{m1} & a_{m2} & ...    & a_{mn} \\
\end{block}
\end{blockarray}
 \]

For instance, in order to craft $1$ unit of product $x_1$ we need to use $a_{11}$ units of input $s_1$, $a_{21}$ units of input $s_2$, and so on. We assume technical coefficients $a_{ij}$ are fixed. Recall that since not all inputs are required for crafting a particular product some of the elements in this matrix may be $0$. Finally, we assume that only outputs can be sold freely at fixed prices: $p_1, p_2,...,p_n$.\\
\\
Given those conditions, we can write the seller problem as follows:
\begin{equation}
\begin{aligned}
\max_{x_1,x_2,...,x_n} \quad & p_1x_1 + p_2x_2 + ... + p_nx_n \\
\textrm{s.t.} \quad & a_{11}x_1 + a_{12}x_2 + ... + a_{1n}x_n \leq s_1\\
  &a_{21}x_1 + a_{22}x_2 + ... + a_{2n}x_n \leq s_2\\
  &\cdots \\
  &a_{m1}x_1 + a_{m2}x_2 + ... + a_{mn}x_n \leq s_m\\
  &x_1,x_2,...,x_n\geq0    \\
\end{aligned}
\end{equation}

Intuitively, the seller problem is to choose production quantities $x_1,x_2,...,x_n$ in order to maximize earnings subject to input availability constraints. This resembles the linear programming canonical form. Therefore, we can use almost any mathematical software (e.g. Matlab, R or Python) to solve this problem. In order to apply this model we need information on technical coefficients, input endowments and prices. We provide an example in the next section.

\section{An example}
Suppose a seller is endowed with $98$ units of input 1, $147$ units of input 2, $182$ units of input 3, $157$ units of input 4 and $114$ units of input 5. By using those inputs the seller can craft up to 6 different products, say $x_1,...,x_6$. The input requirements for crafting each product are summarized in the technical matrix shown below:

\[
A =
\begin{blockarray}{ccccccc}
& x_1 & x_2 & x_3 & x_4 & x_5 & x_6 \\
\begin{block}{c(cccccc)}
  s_1 & 5 & 0 & 0 & 0 & 0 & 4  \\
  s_2 & 0 & 5 & 0 & 0 & 0 & 4  \\
  s_3 & 0 & 0 & 5 & 0 & 3 & 4  \\
  s_4 & 0 & 0 & 0 & 5 & 3 & 4  \\
  s_5 & 0 & 0 & 0 & 0 & 1 & 3  \\
\end{block}
\end{blockarray}
 \]

Finally, market prices for products 1 to 6 are $100.2$, $80$, $110.96$, $120.11$, $115.90$ and $314$ respectively. According to the basic framework described in the previous section we formulate the seller problem for this particular application as follows:

\begin{equation}
\begin{aligned}
\max_{x_1,x_2,...,x_n} \quad & 100.2x_1 + 80x_2 + 110.96x_3 + 120.11x_4 + 115.90x_5 + 314x_6 \\
\textrm{s.t.} \quad & 5x_1 + 4x_6 \leq 98\\
  & 5x_2 + 4x_6 \leq 147\\
  & 5x_3 + 3x_5 + 4x_6 \leq 182\\
  & 5x_4 + 3x_5 + 4x_6 \leq 157\\
  & x_5 + 3x_6 \leq 114\\
  & x_1,x_2,x_3,x_4,x_5,x_6 \geq 0
\end{aligned}
\end{equation}

We use the package \texttt{lpSolve} (linear programming) in the R environment to solve this problem. As shown below, our code\footnote{See the seller.R script in GitHub.} provides us with optimal quantities for every product. Notice that quantity for product $x_5$ is null, this is an example of \textit{corner solution} which is very common in the linear programming framework. This solution implies that, at faced prices, endowments and current technical requirements, is not profitable to craft product $x_5$ at all. 
\begin{Schunk}
\begin{Soutput}
     product quantity
[1,] "x1"    "17"    
[2,] "x2"    "27"    
[3,] "x3"    "34"    
[4,] "x4"    "29"    
[5,] "x5"    "0"     
[6,] "x6"    "3"     
\end{Soutput}
\end{Schunk}

\newpage

\section{Some extentions}
Insofar we have described a basic framework of the seller problem. Under this model we assumed that inputs cannot be sold in the market like products, also we 


\end{document}
